\documentclass[a4paper, 10pt]{article}
\usepackage[utf8]{inputenc}
\usepackage[T1]{fontenc}
\usepackage{polski}
\usepackage{fancyhdr}
\usepackage{enumitem}
\usepackage{lastpage}
\usepackage{tabularx}
\usepackage{xcolor}
\usepackage[colorlinks=true]{hyperref}
\AtBeginDocument{
    \hypersetup{
        urlcolor=blue,
        urlbordercolor=blue,
        pdfborder={1 1 1},
        pdfborderstyle={/S/U/W 1}
    }
}

% pagestyle
\pagestyle{fancy}
\fancyhead{}
\fancyhead[R]{Strona dla lokalnej gazetki}
\fancyfoot{}
\fancyfoot[C]{Strona \thepage\ z \pageref*{LastPage}}

\setlist[itemize,2]{label=$\circ$}

% title data
\title{Zakres projektu i szczegółowe założenia do strony dla lokalnej gazetki}
\author{Szymon Wasążnik}
% \date{}

% document
\begin{document}
\maketitle
\thispagestyle{empty}

\newpage
\thispagestyle{empty}
\hspace{0pt}
\vfill
\begin{abstract}
Dokument ten opisuje funkcjonalności dostępne w końcowym produkcie, narzędzia oraz języki programowania z których pomocą zostanie on wykonany.
\end{abstract}
\vfill
\hspace{0pt}

\newpage
\setcounter{page}{1}
\tableofcontents

\newpage
\section{Słownik}
\begin{itemize}
\setlength{\itemsep}{0pt}
\setlength{\parskip}{0pt}
    \item Strona - serwis WWW służący do zarządzania artykułami
    \item Artkuł - tekst napisany przez użytkownika strony przeznaczony do publikacji
    \item Recenzja - tekst napisany odnośnie konkretnego artykułu (twórca recenzji musi być inny niż artykułu) wraz z oceną w celu ułatwienia decyzji autorowi artykułu o publikacji, edycji lub jego archiwizacji
    \item Użytkownik - uwierzytelniony uczestnik strony
    \item Gość - uczestnik strony nie posiadający konta na stronie
    \item Administrator - uwierzytelniony uczestnik strony, posiadający wyższe uprawnienia
    \item Ocena - liczba całkowita włącznie od 1 do 5
\end{itemize}
\section{Use case}
\subsection{Gość}
\begin{itemize}
\setlength{\itemsep}{0pt}
\setlength{\parskip}{0pt}
    \item Przeglądanie artykułów:
        \begin{itemize}
        \setlength{\itemsep}{0pt}
        \setlength{\parskip}{0pt}
            \item Gość prosi o wyświetlenie artykułów\\
                System odpowiada zwracając tabelę z artykułami, ich autorami i datami: publikacji oraz ostatniej edycji.
            \item Gość prosi o wyświetlenie treści artykułu\\
                System odpowiada zwracając treść artykułu, jego autora oraz daty: publikacji oraz ostatniej edycji.
        \end{itemize}
\end{itemize}
\subsection{Użytkownik}
\begin{itemize}
\setlength{\itemsep}{0pt}
\setlength{\parskip}{0pt}
    \item Dodawanie artykułów:
        \begin{itemize}
        \setlength{\itemsep}{0pt}
        \setlength{\parskip}{0pt}
            \item Użytkownik chce dodać artykuł\\
                Użytkownik przesyła tytuł i treść artykułu\\
                System dodaje artykuł i wyświetla go
        \end{itemize}
    \item Przeglądanie artykułów:
        \begin{itemize}
        \setlength{\itemsep}{0pt}
        \setlength{\parskip}{0pt}
            \item Użytkownik prosi o wyświetlenie artykułów\\
                System odpowiada zwracając tabelę z niezarchiwizowanymi artykułami, ich autorami i datami: stworzenia, ostatniej edycji oraz publikacji.
            \item Użytkownik prosi o wyświetlenie swoich artykułów\\
                System odpowiada zwracając tabelę z artykułami tego użytkownika i ich informacją czy są zarchiwizowane, datami: stworzenia, ostatniej edycji oraz publikacji.
            \item Użytkownik prosi o wyświetlenie artykułu innego użytkownika\\
                System odpowiada zwracając treść artykułu wraz z średnią oceną, autorem oraz datami: stworzenia, ostatniej edycji oraz publikacji.
            \item Użytkownik prosi o wyświetlenie swojego artykułu\\
                System odpowiada zwracając treść artykułu wraz z średnią oceną, recenzjami oraz datami: stworzenia, ostatniej edycji oraz publikacji.
        \end{itemize}
    \item Edytowanie artykułów:
        \begin{itemize}
        \setlength{\itemsep}{0pt}
        \setlength{\parskip}{0pt}
            \item Użytkownik chce edytować swój artykuł\\
                System odpowiada zwracając treść\\
                Użytkownik po edytowaniu przesyła nową treść artykułu\\
                System aktualizuje artykuł i wyświetla artykuł po edycji
            \item Użytkownik chce edytować artykuł innego użytkownika\\
                System zwraca informację o braku uprawnień
        \end{itemize}
    \item Archiwizowanie artykułów:
        \begin{itemize}
        \setlength{\itemsep}{0pt}
        \setlength{\parskip}{0pt}
            \item Użytkownik chce zarchiwizować swój artykuł\\
                System archiwizuje artykuł
            \item Użytkownik chce zarchiwizować artykuł innego użytkownika\\
                System zwraca informację o braku uprawnień
        \end{itemize}
    \item Dearchiwizowanie artykułów:
        \begin{itemize}
        \setlength{\itemsep}{0pt}
        \setlength{\parskip}{0pt}
            \item Użytkownik chce dearchiwizować swój artykuł\\
                System dearchiwizuje artykuł
        \end{itemize}
    \item Dodawanie recenzji:
        \begin{itemize}
        \setlength{\itemsep}{0pt}
        \setlength{\parskip}{0pt}
            \item Użytkownik chce dodać recenzję do swojego artykułu\\
                Użytkownik przesyła treść recenzji i informację do którego artykułu została stworzona\\
                System zwraca informację o braku uprawnień
            \item Użytkownik chce dodać recenzję do artykułu innego użytkownika\\
                Użytkownik przesyła treść recenzji i informację do którego artykułu została stworzona\\
                System dodaje recenzję i wyświetla ją
            \item Użytkownik chce dodać recenzję do tego samego artykułu po raz drugi (bez usunięcia poprzedniej)\\
                System zwraca informację o niepoprawnej operacji
        \end{itemize}
    \item Przeglądanie recenzji:
        \begin{itemize}
        \setlength{\itemsep}{0pt}
        \setlength{\parskip}{0pt}
            \item Użytkownik prosi o wyświetlenie recenzji (l.mn.) swojego artykułu\\
                System odpowiada zwracając tabelę z recenzjami, ich autorami i ocenami
            \item Użytkownik prosi o wyświetlenie recenzji (l.p.) swojego artykułu\\
                System odpowiada zwracając treść recenzji, jej autora i oceną
            \item Użytkownik prosi o wyświetlenie swoich recenzji\\
                System odpowiada zwracając tabelę z recenzjami, ich ocenami i artykułami do których zostały stworzone
            \item Użytkownik prosi o wyświetlenie swojej recenzji\\
                System odpowiada zwracając treść recenzji, jej oceną i artykułem do którego został stworzony
        \end{itemize}
    \item Edytowanie recenzji:
        \begin{itemize}
        \setlength{\itemsep}{0pt}
        \setlength{\parskip}{0pt}
            \item Użytkownik chce edytować swoją recenzję\\
                System odpowiada zwracając treść\\
                Użytkownik po edytowaniu przesyła nową treść recenzji\\
                System aktualizuje recenzję i wyświetla ją po edycji
            \item Użytkownik chce edytować recenzję innego użytkownika\\
                System zwraca informację o braku uprawnień
        \end{itemize}
    \item Usuwanie recenzji:
        \begin{itemize}
        \setlength{\itemsep}{0pt}
        \setlength{\parskip}{0pt}
            \item Użytkownik chce usunąć swoją recenzję\\
                System usuwa recenzję
            \item Użytkownik chce usunąć recenzję innego użytkownika\\
                System zwraca informację o braku uprawnień
        \end{itemize}
\end{itemize}
\subsection{Administrator}
\begin{itemize}
\setlength{\itemsep}{0pt}
\setlength{\parskip}{0pt}
    \item Dodawanie użytkowników:
        \begin{itemize}
        \setlength{\itemsep}{0pt}
        \setlength{\parskip}{0pt}
            \item Administrator prosi o wyświetlenie artykułów\\
                System odpowiada zwracając tabelę z artykułami, ich autorami i datami: stworzenia, ostatniej edycji oraz publikacji
        \end{itemize}
    \item Przeglądanie użytkowników:
        \begin{itemize}
        \setlength{\itemsep}{0pt}
        \setlength{\parskip}{0pt}
            \item Administrator prosi o wyświetlenie użytkowników\\
                System odpowiada zwracając tabelę z użytkownikami i informacją czy są administratorami
        \end{itemize}
    \item Archiwizowanie użytkowników:
        \begin{itemize}
        \setlength{\itemsep}{0pt}
        \setlength{\parskip}{0pt}
            \item Administrator prosi o zarchiwizowanie użytkownika nie będącego administratorem\\
                System archiwizuje użytkownika
            \item Administrator prosi o zarchiwizowanie użytkownika będącego administratorem\\
                System zwraca informację o braku uprawnień
        \end{itemize}
    \item Dearchiwizowanie użytkowników:
        \begin{itemize}
        \setlength{\itemsep}{0pt}
        \setlength{\parskip}{0pt}
            \item Administrator prosi o dearchiwizowanie użytkownika\\
                System dearchiwizuje użytkownika
        \end{itemize}
    \item Usuwanie uprawnienia administratora użytkownikom:
        \begin{itemize}
        \setlength{\itemsep}{0pt}
        \setlength{\parskip}{0pt}
            \item Administrator prosi o usunięcie sobie uprawnienia\\
                System zwraca informację o braku uprawnień
            \item Administrator prosi o usunięcie uprawnienia innemu użytkownikowi\\
                System zwraca informację o braku uprawnień
        \end{itemize}
\end{itemize}
\section{Narzędzia i języki}
\begin{itemize}
\setlength{\itemsep}{0pt}
\setlength{\parskip}{0pt}
    \item Języki wykonywane po stronie serwera:\\
        PHP, Bourne Shell, PSQL, YAML
    \item Narzędzia użyte po stronie serwera:\\
        Docker-Compose, Docker
    \item Narzędzia użyte po stronie serwera w kontenerach:\\
        php, sh, postgresql, nginx
    \item Języki wykonywane po stronie uczestnika serwisu:\\
        JavaScript, HTML, CSS
    \item Narzędzia użyte po stronie uczestnika serwisu:\\
        przeglądarka internetowa
\end{itemize}
\end{document}
