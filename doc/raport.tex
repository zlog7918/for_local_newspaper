\documentclass[a4paper, 10pt]{article}
\usepackage[utf8]{inputenc}
\usepackage[T1]{fontenc}
\usepackage{polski}
\usepackage{fancyhdr}
\usepackage{enumitem}
\usepackage{lastpage}
\usepackage{tabularx}
\usepackage{xcolor}
\usepackage{graphicx}
\usepackage[colorlinks=true]{hyperref}
\newcommand*{\fullref}[1]{\hyperref[{#1}]{\ref*{#1} \nameref*{#1}}}
\AtBeginDocument{
    \hypersetup{
        urlcolor=blue,
        urlbordercolor=blue,
        pdfborder={1 1 1},
        pdfborderstyle={/S/U/W 1}
    }
}

% pagestyle
\pagestyle{fancy}
\fancyhead{}
\fancyhead[R]{Strona dla lokalnej gazetki}
\fancyfoot{}
\fancyfoot[C]{Strona \thepage\ z \pageref*{LastPage}}

\setlist[itemize,2]{label=$\circ$}

% title data
\title{Zakres projektu i podsumowanie}
\author{Szymon Wasążnik}
% \date{}

% document
\begin{document}
\maketitle
\thispagestyle{empty}

\newpage
\thispagestyle{empty}
\hspace{0pt}
\vfill
\begin{abstract}
Dokument ten opisuje funkcjonalności dostępne w końcowym produkcie, narzędzia oraz języki programowania z których pomocą został on wykonany.
\end{abstract}
\vfill
\hspace{0pt}

\newpage
\setcounter{page}{1}
\tableofcontents

\newpage
\section{Zrealizowane {\tt use case}{\textquotesingle}y}
\subsection{Gość}
\begin{itemize}
\setlength{\itemsep}{0pt}
\setlength{\parskip}{0pt}
    \item Przeglądanie artykułów:
        \begin{itemize}
        \setlength{\itemsep}{0pt}
        \setlength{\parskip}{0pt}
            \item Gość prosi o wyświetlenie artykułów\\
                \includegraphics[width=\textwidth]{gosc_get_articles.png}
            \item Gość prosi o wyświetlenie treści artykułu\\
                Po wybraniu artykułu:\\
                \includegraphics[width=\textwidth]{gosc_view_article.png}
        \end{itemize}
\end{itemize}
\newpage
\subsection{Użytkownik}
\begin{itemize}
\setlength{\itemsep}{0pt}
\setlength{\parskip}{0pt}
    \item Dodawanie artykułów:
        \begin{itemize}
        \setlength{\itemsep}{0pt}
        \setlength{\parskip}{0pt}
            \item Użytkownik chce dodać artykuł\\
                \includegraphics[width=\textwidth]{user_create_article.png}
        \end{itemize}
    \item Przeglądanie artykułów:
        \begin{itemize}
        \setlength{\itemsep}{0pt}
        \setlength{\parskip}{0pt}
            \item Użytkownik prosi o wyświetlenie artykułów
            \item Użytkownik prosi o wyświetlenie swoich artykułów
            \item Użytkownik prosi o wyświetlenie artykułu innego użytkownika
            \item Użytkownik prosi o wyświetlenie swojego artykułu
        \end{itemize}
    \item Edytowanie artykułów:
        \begin{itemize}
        \setlength{\itemsep}{0pt}
        \setlength{\parskip}{0pt}
            \item Użytkownik chce edytować swój artykuł
            \item Użytkownik chce edytować artykuł innego użytkownika
        \end{itemize}
    \item Archiwizowanie artykułów:
        \begin{itemize}
        \setlength{\itemsep}{0pt}
        \setlength{\parskip}{0pt}
            \item Użytkownik chce zarchiwizować swój artykuł
            \item Użytkownik chce zarchiwizować artykuł innego użytkownika
        \end{itemize}
    \item Dodawanie recenzji:
        \begin{itemize}
        \setlength{\itemsep}{0pt}
        \setlength{\parskip}{0pt}
            \item Użytkownik chce dodać recenzję do swojego artykułu
            \item Użytkownik chce dodać recenzję do artykułu innego użytkownika
            \item Użytkownik chce dodać recenzję do tego samego artykułu po raz drugi (bez usunięcia poprzedniej)
        \end{itemize}
    \item Przeglądanie recenzji:
        \begin{itemize}
        \setlength{\itemsep}{0pt}
        \setlength{\parskip}{0pt}
            \item Użytkownik prosi o wyświetlenie recenzji (l.mn.) swojego artykułu
            \item Użytkownik prosi o wyświetlenie recenzji (l.p.) swojego artykułu
            \item Użytkownik prosi o wyświetlenie swoich recenzji
            \item Użytkownik prosi o wyświetlenie swojej recenzji
        \end{itemize}
    \item Edytowanie recenzji:
        \begin{itemize}
        \setlength{\itemsep}{0pt}
        \setlength{\parskip}{0pt}
            \item Użytkownik chce edytować swoją recenzję
            \item Użytkownik chce edytować recenzję innego użytkownika
        \end{itemize}
    \item Usuwanie recenzji:
        \begin{itemize}
        \setlength{\itemsep}{0pt}
        \setlength{\parskip}{0pt}
            \item Użytkownik chce usunąć swoją recenzję
            \item Użytkownik chce usunąć recenzję innego użytkownika
        \end{itemize}
    \item Dearchiwizowanie artykułów:
        \begin{itemize}
        \setlength{\itemsep}{0pt}
        \setlength{\parskip}{0pt}
            \item Użytkownik chce dearchiwizować swój artykuł\\
                Po use case {\tt Przeglądanie artykułów} i wybraniu artykułu:\\
                \includegraphics[width=\textwidth]{user_before_dearchive.png}
                Po wciśnięciu zaznaczonego przycisku artykuł został dearchiwizowany:\\
                \includegraphics[width=\textwidth]{user_after_dearchive.png}
        \end{itemize}
\end{itemize}
% \newpage
\subsection{Administrator}
\begin{itemize}
\setlength{\itemsep}{0pt}
\setlength{\parskip}{0pt}
    \item Dodawanie użytkowników:
        \begin{itemize}
        \setlength{\itemsep}{0pt}
        \setlength{\parskip}{0pt}
            \item Administrator prosi o dodanie użytkownika
        \end{itemize}
\newpage
    \item Przeglądanie użytkowników:
        \begin{itemize}
        \setlength{\itemsep}{0pt}
        \setlength{\parskip}{0pt}
            \item Administrator prosi o wyświetlenie użytkowników\\
                \includegraphics[width=\textwidth]{admin_create_user.png}
        \end{itemize}
    \item Archiwizowanie użytkowników:
        \begin{itemize}
        \setlength{\itemsep}{0pt}
        \setlength{\parskip}{0pt}
            \item Administrator prosi o zarchiwizowanie użytkownika nie będącego administratorem
            \item Administrator prosi o zarchiwizowanie użytkownika będącego administratorem
        \end{itemize}
    \item Dearchiwizowanie użytkowników:
        \begin{itemize}
        \setlength{\itemsep}{0pt}
        \setlength{\parskip}{0pt}
            \item Administrator prosi o dearchiwizowanie użytkownika
        \end{itemize}
    \item Usuwanie uprawnienia administratora użytkownikom:
        \begin{itemize}
        \setlength{\itemsep}{0pt}
        \setlength{\parskip}{0pt}
            \item Administrator prosi o usunięcie sobie uprawnienia
            \item Administrator prosi o usunięcie uprawnienia innemu użytkownikowi
        \end{itemize}
\end{itemize}
\section{Niezrealizowane aspekty}
\begin{itemize}
\setlength{\itemsep}{0pt}
\setlength{\parskip}{0pt}
    \item Brak sposobów filtrowania danych
\end{itemize}
\section{Narzędzia i języki, którymi zrealizowano projekt}
\begin{itemize}
\setlength{\itemsep}{0pt}
\setlength{\parskip}{0pt}
    \item Języki wykonywane po stronie serwera:\\
        PHP, PSQL, YAML
    \item Narzędzia użyte po stronie serwera:\\
        Docker-Compose, Docker
    \item Narzędzia użyte po stronie serwera w kontenerach:\\
        php, postgresql, nginx
    \item Języki wykonywane po stronie uczestnika serwisu:\\
        JavaScript, HTML, CSS
    \item Narzędzia użyte po stronie uczestnika serwisu:\\
        przeglądarka internetowa
\end{itemize}
\section{Podsumowanie}
Projekt został wykonany obejmując wszystkie podstawowe funkcjonalności.\\
Następnymi krokami przy rozwijaniu tego projektu byłoby:
\begin{itemize}
\setlength{\itemsep}{0pt}
\setlength{\parskip}{0pt}
    \item Dodanie filtrowania danych
    \item Poprawienie szaty graficznej interfejsu użytkownika
    \item Zaimplementowanie bardziej zaawansowanej walidacji danych i ujednolicenie obsługi błędów
\end{itemize}
\end{document}
